\documentclass[10pt]{article}
\usepackage{graphicx}
\def\inputGnumericTable{}
\usepackage[latin1]{inputenc}
\usepackage{fullpage}
\usepackage{color}
\usepackage{array}
\usepackage{longtable}
\usepackage{calc}
\usepackage{multirow}
\usepackage{hhline}
\usepackage{ifthen}
\usepackage{amsmath}
\usepackage[none]{hyphenat}
\usepackage{listings}
\usepackage[english]{babel}
\usepackage{siunitx}
\usepackage{caption}
\usepackage{booktabs}
\usepackage{array}
\usepackage{extarrows}
\usepackage{enumerate}
\usepackage{enumitem}
\usepackage{amsmath}
\usepackage{commath}
\usepackage{gensymb}
\usepackage{amssymb}
\usepackage{multicol}
%\usepackage[utf8]{inputenc}
\lstset{
 frame=single,
 breaklines=true
}
\usepackage{hyperref}
\usepackage[margin=0.5in]{geometry}	 
%\usepackage{exsheets}% also loads the `tasks' package
\usepackage{atbegshi}
\AtBeginDocument{\AtBeginShipoutNext{\AtBeginShipoutDiscard}}

%new macro definitions
\renewcommand{\labelenumi}{(\roman{enumi})}
\newcommand{\mydet}[1]{\ensuremath{\begin{vmatrix}#1\end{vmatrix}}}
\providecommand{\brak}[1]{\ensuremath{\left(#1\right)}}
\newcommand{\solution}{\noindent \textbf{Solution: }}
\newcommand{\myvec}[1]{\ensuremath{\begin{pmatrix}#1\end{pmatrix}}}
\newenvironment{amatrix}[1]{%
	\left(\begin{array}{@{}*{#1}{c}|c@{}}
}{%
	\end{array}\right)
}

\newcommand{\myaugvec}[2]{\ensuremath{\begin{amatrix}{#1}#2\end{amatrix}}}
\providecommand{\norm}[1]{\left\1Vert#1\right\rVert}
\let\vec\mathbf{}


%\SetEnumitemKey{twocol}{
% before=\raggedcolumns\begin{multicols}{2},
% after=\end{multicols}}
%\SetEnumitemKey{fourcol}{
% before=\raggedcolumns\begin{multicols}{4},
% after=\end{multicols}} 


\begin{document}
\begin{center}
\title{\textbf{TRIANGLES}}
\date{\vspace{-5ex}}
\maketitle
\end{center}
\section*{9$^{th}$Math - Chapter 7}
\section*{Problem}
In the given Figure \ref{fig:1}, AC=AE,AB=AD and $\angle BAD=\angle EAC$.Show that BC=DE.
\begin{figure}[!h]
	\begin{center}
	\includegraphics[width=\columnwidth]{./figs/fig.pdf}
	\end{center}
\caption{}
\label{fig:1}
\end{figure}
\section*{Construction}
The input parameters for construction.\\

\begin{table}[!h]
\centering
\input{./table/table.tex}
\label{tab:1}
\end{table}
\begin{align}
b&=\sqrt{a^2+c^2-2ac\cos\theta}\\
\vec{A}=c\myvec{\cos\theta\\\sin\theta},
\vec{B}&=\myvec{0\\0},
\vec{C}=\myvec{a\\0},
\vec{D}=\brak{2c\sin\frac{\theta}{2}}\vec{e_1}\\
\angle BCA&=cos^{-1}\brak{\frac{a^2+b^2-c^2}{2ab}}\\
\angle ACE&=90\degree-\frac{\theta}{2}\\
\phi&=180\degree-\brak{\angle BCA+\angle ACE}\\
\vec{E}&=\vec{C}+\brak{2b\sin\frac{\theta}{2}}\myvec{\cos\phi\\\sin\phi}\\
\end{align}
\solution
\textbf{Given:}
\begin{align}
    \vec{A}-\vec{C}&=\vec{A}-\vec{E}\\
    \vec{A}-\vec{B}&=\vec{A}-\vec{D}\\
    \angle{BAD}&=\angle{EAC}    
\end{align}
\textbf{To prove :}
\begin{align}
    \vec{B}-\vec{C}=\vec{D}-\vec{E}
\end{align}
\textbf{Proof}\\
In $\triangle ABC $ and in $\triangle ADE$
\begin{align}
\norm{\vec{A}-\vec{B}}=&\norm{\myvec{2.5\\4.33}}=5\\
\norm{\vec{A}-\vec{D}}=&\norm{\myvec{-2.6\\4.33}}=5\\
\implies\norm{\vec{A}-\vec{B}}=&\norm{\vec{A}-\vec{D}}
\label{eq:1}\\
\norm{\vec{A}-\vec{C}}=&\norm{\myvec{-5.5\\4.33}}=7\\
\norm{\vec{A}-\vec{E}}=&\norm{\myvec{-6.5\\-2.5}}=7\\
\implies\norm{\vec{A}-\vec{C}}=&\norm{\vec{A}-\vec{E}}
\label{eq:2}\\
\angle BAC=&\cos^{-1}\frac{\brak{\vec{A}-\vec{B}}^{\top}\brak{\vec{A}-\vec{C}}}{\norm{\vec{A}-\vec{B}}\norm{\vec{A}-\vec{C}}}=82\degree
\label{eq:3}\\
\angle DAE=&\cos^{-1}\frac{\brak{\vec{A}-\vec{D}}^{\top}\brak{\vec{A}-\vec{E}}}{\norm{\vec{A}-\vec{D}}\norm{\vec{A}-\vec{E}}}=82\degree
\label{eq:4}
\end{align}
from $\eqref{eq:3}$ and $\eqref{eq:4}$
\begin{align}
\angle BAC=&\angle DAE
\label{eq:5}
\end{align}
from $\eqref{eq:1}$,$\eqref{eq:2}$ and $\eqref{eq:5}$\\
\begin{align}
\triangle{ABC} \cong \triangle{ADE}
\label{eq:6}
\end{align} 
from $\eqref{eq:6}$\\
\begin{align}   
\vec{B}-\vec{C} &= \vec{D}-\vec{E}
\end{align}
\end{document}


