\documentclass[10pt]{article}
\usepackage{graphicx}
\usepackage[none]{hyphenat}
\usepackage{graphicx}
\usepackage{listings}
\usepackage[english]{babel}
\usepackage{siunitx}
\usepackage{graphicx}
\usepackage{caption} 
\usepackage{booktabs}
\usepackage{array}
\usepackage{amssymb} % for \because
\usepackage{amsmath}   % for having text in math mode
\usepackage{extarrows} % for Row operations arrows
\usepackage{listings}
\usepackage[utf8]{inputenc}
\lstset{
  frame=single,
  breaklines=true
}
\usepackage{hyperref}
  
%Following 2 lines were added to remove the blank page at the beginning
\usepackage{atbegshi}% http://ctan.org/pkg/atbegshi
\AtBeginDocument{\AtBeginShipoutNext{\AtBeginShipoutDiscard}}


%New macro definitions
\newcommand{\mydet}[1]{\ensuremath{\begin{vmatrix}#1\end{vmatrix}}}
\providecommand{\brak}[1]{\ensuremath{\left(#1\right)}}
\newcommand{\solution}{\noindent \textbf{Solution: }}
\newcommand{\myvec}[1]{\ensuremath{\begin{pmatrix}#1\end{pmatrix}}}
\providecommand{\norm}[1]{\left\lVert#1\right\rVert}
\providecommand{\abs}[1]{\left\vert#1\right\vert}
\let\vec\mathbf{}
\begin{document}

\begin{center}
\title{\textbf{VECTORS}}
\date{\vspace{-5ex}} %Not to print date automatically
\maketitle
\end{center}

\section{12$^{th}$ Maths - EXERCISE-10.3}

\begin{enumerate}
\item Find $\abs{\overrightarrow{a}}$ and $\abs{\overrightarrow{b}}$, if ($\overrightarrow{a}$+$\overrightarrow{b}$)$.$($\overrightarrow{a}$-$\overrightarrow{b}$)=8 and ${\overrightarrow{a}}$=8$\abs{\overrightarrow{b}}$.\\  

\solution
Given  points are

\begin{align}
(\overrightarrow{a}+\overrightarrow{b}).(\overrightarrow{a}-\overrightarrow{b})&=8
\end{align}
\begin{align}
\abs{\overrightarrow{a}}&=8\abs{\overrightarrow{b}}
\end{align}
\begin{align}
(\overrightarrow{a}+\overrightarrow{b}).(\overrightarrow{a}-\overrightarrow{b})&=8
\end{align}
\begin{align}
\overrightarrow{a}.(\overrightarrow{a}-\overrightarrow{b})+\overrightarrow{b}.(\overrightarrow{a}-\overrightarrow{b})&=8
\end{align}
\begin{align}
\overrightarrow{a}.\overrightarrow{a}-\overrightarrow{a}.\overrightarrow{b}+\overrightarrow{b}.\overrightarrow{a}-\overrightarrow{b}.\overrightarrow{b}&=8
\end{align}
\begin{align}
\overrightarrow{a}.\overrightarrow{a}-\overrightarrow{a}.\overrightarrow{b}+\overrightarrow{a}.\overrightarrow{b}-\overrightarrow{b}.\overrightarrow{b}&=8
\end{align}
\begin{align}
\overrightarrow{a}.\overrightarrow{a}-\overrightarrow{b}.\overrightarrow{b}&=8
\end{align}
\begin{align}
(\abs{\overrightarrow{a}})^2-(\abs{\overrightarrow{b}})^2&=8
\end{align}
\begin{align}
(\abs{8\overrightarrow{b}})^2-(\abs{\overrightarrow{b}})^2&=8
\end{align}
\begin{align}
64{\overrightarrow{b}}^2-\overrightarrow{b}^2&=8
\end{align}
\begin{align}
63{\overrightarrow{b}}^2&=8
\end{align}
\begin{align}
\overrightarrow{b}^2&=\frac{8}{63}
\end{align}
\begin{align}
\overrightarrow{b}&=\sqrt{\frac{8}{63}}
\end{align}
\begin{align}
\overrightarrow{b}&=\frac{2\sqrt{2}}{3\sqrt{7}}
\end{align}
\begin{align}
\abs{\overrightarrow{a}}&=8\abs{\overrightarrow{b}}
\end{align}
\begin{align}
\abs{\overrightarrow{a}}&=8.\frac{2\sqrt{2}}{3\sqrt{7}}
\end{align}
\begin{align}
\abs{\overrightarrow{a}}&=\frac{16\sqrt{2}}{3\sqrt{7}}
\end{align}
\end{enumerate}
\end{document}
