\documentclass[10pt]{article}
\usepackage{graphicx}
\def\inputGnumericTable{}
\usepackage[latin1]{inputenc}
\usepackage{fullpage}
\usepackage{color}
\usepackage{array}
\usepackage{longtable}
\usepackage{calc}
\usepackage{multirow}
\usepackage{hhline}
\usepackage{ifthen}
\usepackage{amsmath}
\usepackage[none]{hyphenat}
\usepackage{listings}
\usepackage[english]{babel}
\usepackage{siunitx}
\usepackage{caption}
\usepackage{booktabs}
\usepackage{array}
\usepackage{extarrows}
\usepackage{enumerate}
\usepackage{enumitem}
\usepackage{amsmath}
\usepackage{commath}
\usepackage{gensymb}
\usepackage{amssymb}
\usepackage{multicol}
%\usepackage[utf8]{inputenc}
\lstset{
 frame=single,
 breaklines=true
}
\usepackage{hyperref}
\usepackage[margin=0.5in]{geometry}	 
%\usepackage{exsheets}% also loads the `tasks' package
\usepackage{atbegshi}
\AtBeginDocument{\AtBeginShipoutNext{\AtBeginShipoutDiscard}}

%new macro definitions
\renewcommand{\labelenumi}{(\roman{enumi})}
\newcommand{\mydet}[1]{\ensuremath{\begin{vmatrix}#1\end{vmatrix}}}
\providecommand{\brak}[1]{\ensuremath{\left(#1\right)}}
\newcommand{\solution}{\noindent \textbf{Solution: }}
\newcommand{\myvec}[1]{\ensuremath{\begin{pmatrix}#1\end{pmatrix}}}
\newenvironment{amatrix}[1]{%
	\left(\begin{array}{@{}*{#1}{c}|c@{}}
}{%
	\end{array}\right)
}

\newcommand{\myaugvec}[2]{\ensuremath{\begin{amatrix}{#1}#2\end{amatrix}}}
\providecommand{\norm}[1]{\left\1Vert#1\right\rVert}
\let\vec\mathbf{}


%\SetEnumitemKey{twocol}{
% before=\raggedcolumns\begin{multicols}{2},
% after=\end{multicols}}
%\SetEnumitemKey{fourcol}{
% before=\raggedcolumns\begin{multicols}{4},
% after=\end{multicols}} 


\begin{document}
\begin{center}
\title{\textbf{CIRCLES}}
\date{\vspace{-5ex}}
\maketitle
\end{center}
\section*{9$^{th}$Math - Chapter 10}
\section*{Problem}
If the  non-parallel sides of a trapezium are equal,prove that it is cyclic.
\begin{figure}[!h]
	\begin{center}
	\includegraphics[width=\columnwidth]{./figs/fig.pdf}
	\end{center}
\caption{}
\label{fig:1}
\end{figure}
\section*{Construction}
The input parameters for construction.\\

\begin{table}[!h]
\centering
\input{./table/table.tex}
\label{tab:1}
\end{table}
\begin{align}
\vec{A}=r\myvec{\cos\theta_2\\\sin\theta_2},
\vec{B}=&r\myvec{\cos\theta_3\\\sin\theta_3},
\vec{C}=r\myvec{\cos\theta_1\\\sin\theta_1},
\vec{D}=r\myvec{-\cos\theta_1\\\sin\theta_1}\\
\vec{P_1}=&b\myvec{\cos\theta_2\\\sin\theta_2},
\vec{P_2}=b\myvec{\cos\theta_3\\\sin\theta_3}
\end{align}
\solution\\
\textbf{Theorm:}
If the sum of a pair of opposite angles of a quadrilateral is
180$\degree$, the quadrilateral is cyclic.\\

\textbf{Proof:}\\
\begin{align}
\angle DAP_1=&\cos^{-1}\frac{\brak{\vec{A}-\vec{D}}^{\top}\brak{\vec{A}-\vec{P_1}}}{\norm{\vec{A}-\vec{D}}\norm{\vec{A}-\vec{P_1}}}=67.5\degree
\label{eq:1}\\
\angle DCP_2=&\cos^{-1}\frac{\brak{\vec{C}-\vec{D}}^{\top}\brak{\vec{P_2}-\vec{C}}}{\norm{\vec{C}-\vec{D}}\norm{\vec{P_2}-\vec{C}}}=90\degree
\label{eq:2}\\
\angle BCP_2=&\cos^{-1}\frac{\brak{\vec{B}-\vec{C}}^{\top}\brak{\vec{P_2}-\vec{C}}}{\norm{\vec{B}-\vec{C}}\norm{\vec{P_2}-\vec{C}}}=22.5\degree
\label{eq:3}
\end{align}
from $\eqref{eq:1}$,$\eqref{eq:2}$ and $\eqref{eq:3}$
\begin{align}
\angle DAP_1+\angle DCP_2+\angle BCP_2=180\degree
\label{eq:4}
\end{align}
from $\eqref{eq:4}$
given quadrilateral is cyclic.
\end{document}


